\documentclass{uist96}
\usepackage{times,epic,eepic}


\newcommand{\code}{\small\sf}
% \newcommand{\code}{\sf}
\newcommand{\finish}[1]{{\bf #1}\marginpar{\Large $\bullet$}}

% don't do anything special with strengths this time
\newcommand{\strength}{\rm}
%\newcommand{\strength}{\bf}

%%%\input{abb1.tex}
\newcommand{\bs}{\begin{slide}{}}
\newcommand{\es}{\end{slide}}
\newcommand{\be}{\begin{eqnarray*}}
\newcommand{\ee}{\end{eqnarray*}}
\newcommand{\bt}{\begin{tabular}}
\newcommand{\et}{\end{tabular}}
\newcounter{bean}
\newcommand{\bl}[1]{\begin{list}{#1}{\usecounter{bean}}}
\newcommand{\el}{\end{list}}
\newcommand{\bel}[1]{\begin{equation} \label{#1}}
\newcommand{\eel}{\end{equation}}
\newenvironment{mat}{\left[\begin{array}}{\end{array}\right]}
\newenvironment{dt}{\left|\begin{array}}{\end{array}\right|}
\newcommand{\bip}{\langle} \newcommand{\eip}{\rangle}
\newenvironment{arr}{\begin{array}}{\end{array}}
\def\x1{x_{1}} \def\x2{x_{2}} \def\x3{x_{3}}
\newcommand{\ebs}{\end{slide}\begin{slide}{}}
\newcommand{\bc}{\begin{center}}
\newcommand{\ec}{\end{center}}
\newcommand{\Lra}{\Leftrightarrow}
\newcommand{\lra}{\leftrightarrow}
\newcommand{\Llra}{\Longleftrightarrow}
\newcommand{\La}{\Leftarrow}
\newcommand{\Ra}{\Rightarrow}
\newcommand{\la}{\leftarrow}
\newcommand{\ra}{\rightarrow}
\newcommand{\ua}{\uparrow}
\newcommand{\da}{\downarrow}
\def\eop{\vspace{5mm} \hfill $\Box$}
\newtheorem{lem}{Lemma}[section]
\newtheorem{thm}{Theorem}[section]
\newtheorem{cor}{Corollary}[section]
%\newfigure{fig}{Figure}[section]
%\newtable{tab}{Table}[section]
\def\sr{{\cal R}} \def\cs{{\cal C}}
\def\va{{\bf a}} \def\vb{{\bf b}} \def\vc{{\bf c}} \def\vd{{\bf d}}
\def\ve{{\bf e}} \def\vf{{\bf f}} \def\vg{{\bf g}} \def\vh{{\bf h}}
\def\vi{{\bf i}} \def\vj{{\bf j}} \def\vk{{\bf k}} \def\vl{{\bf l}}
\def\vm{{\bf m}} \def\vn{{\bf n}} \def\vo{{\bf o}} \def\vp{{\bf p}}
\def\vq{{\bf q}} \def\vr{{\bf r}} \def\vs{{\bf s}} \def\vt{{\bf t}}
\def\vu{{\bf u}} \def\vv{{\bf v}} \def\vw{{\bf w}} \def\vx{{\bf x}}
\def\vy{{\bf y}} \def\vz{{\bf z}} \def\v0{{\bf 0}}
\def\bigf{\nabla f}
\def\bigt{\nabla}
\def\big2f{\nabla^{2}f}
\def\vlambda{{\bf \lambda}}
\def\QED{\ifhmode\unskip\nobreak\fi\ifmmode\ifinner\else\hskip5pt\fi\fi
  \hbox{\hskip5pt\vrule width5pt height5pt depth1.5pt\hskip1pt}}

\newcommand{\bs}{\begin{slide}{}}
\newcommand{\es}{\end{slide}}
\newcommand{\be}{\begin{eqnarray*}}
\newcommand{\ee}{\end{eqnarray*}}
\newcommand{\bt}{\begin{tabular}}
\newcommand{\et}{\end{tabular}}
\newcounter{bean}
\newcommand{\bl}[1]{\begin{list}{#1}{\usecounter{bean}}}
\newcommand{\el}{\end{list}}
\newcommand{\bel}[1]{\begin{equation} \label{#1}}
\newcommand{\eel}{\end{equation}}
\newenvironment{mat}{\left[\begin{array}}{\end{array}\right]}
\newenvironment{dt}{\left|\begin{array}}{\end{array}\right|}
\newcommand{\bip}{\langle} \newcommand{\eip}{\rangle}
\newenvironment{arr}{\begin{array}}{\end{array}}
\def\x1{x_{1}} \def\x2{x_{2}} \def\x3{x_{3}}
\newcommand{\ebs}{\end{slide}\begin{slide}{}}
\newcommand{\bc}{\begin{center}}
\newcommand{\ec}{\end{center}}
\newcommand{\Lra}{\Leftrightarrow}
\newcommand{\lra}{\leftrightarrow}
\newcommand{\Llra}{\Longleftrightarrow}
\newcommand{\La}{\Leftarrow}
\newcommand{\Ra}{\Rightarrow}
\newcommand{\la}{\leftarrow}
\newcommand{\ra}{\rightarrow}
\newcommand{\ua}{\uparrow}
\newcommand{\da}{\downarrow}
\def\eop{\vspace{5mm} \hfill $\Box$}
\newtheorem{lem}{Lemma}[section]
\newtheorem{thm}{Theorem}[section]
\newtheorem{cor}{Corollary}[section]
%\newfigure{fig}{Figure}[section]
%\newtable{tab}{Table}[section]
\def\sr{{\cal R}} \def\cs{{\cal C}}
\def\va{{\bf a}} \def\vb{{\bf b}} \def\vc{{\bf c}} \def\vd{{\bf d}}
\def\ve{{\bf e}} \def\vf{{\bf f}} \def\vg{{\bf g}} \def\vh{{\bf h}}
\def\vi{{\bf i}} \def\vj{{\bf j}} \def\vk{{\bf k}} \def\vl{{\bf l}}
\def\vm{{\bf m}} \def\vn{{\bf n}} \def\vo{{\bf o}} \def\vp{{\bf p}}
\def\vq{{\bf q}} \def\vr{{\bf r}} \def\vs{{\bf s}} \def\vt{{\bf t}}
\def\vu{{\bf u}} \def\vv{{\bf v}} \def\vw{{\bf w}} \def\vx{{\bf x}}
\def\vy{{\bf y}} \def\vz{{\bf z}} \def\v0{{\bf 0}}
\def\bigf{\nabla f}
\def\bigt{\nabla}
\def\big2f{\nabla^{2}f}
\def\vlambda{{\bf \lambda}}
\def\QED{\ifhmode\unskip\nobreak\fi\ifmmode\ifinner\else\hskip5pt\fi\fi
  \hbox{\hskip5pt\vrule width5pt height5pt depth1.5pt\hskip1pt}}

\newcommand{\bs}{\begin{slide}{}}
\newcommand{\es}{\end{slide}}
\newcommand{\be}{\begin{eqnarray*}}
\newcommand{\ee}{\end{eqnarray*}}
\newcommand{\bt}{\begin{tabular}}
\newcommand{\et}{\end{tabular}}
\newcounter{bean}
\newcommand{\bl}[1]{\begin{list}{#1}{\usecounter{bean}}}
\newcommand{\el}{\end{list}}
\newcommand{\bel}[1]{\begin{equation} \label{#1}}
\newcommand{\eel}{\end{equation}}
\newenvironment{mat}{\left[\begin{array}}{\end{array}\right]}
\newenvironment{dt}{\left|\begin{array}}{\end{array}\right|}
\newcommand{\bip}{\langle} \newcommand{\eip}{\rangle}
\newenvironment{arr}{\begin{array}}{\end{array}}
\def\x1{x_{1}} \def\x2{x_{2}} \def\x3{x_{3}}
\newcommand{\ebs}{\end{slide}\begin{slide}{}}
\newcommand{\bc}{\begin{center}}
\newcommand{\ec}{\end{center}}
\newcommand{\Lra}{\Leftrightarrow}
\newcommand{\lra}{\leftrightarrow}
\newcommand{\Llra}{\Longleftrightarrow}
\newcommand{\La}{\Leftarrow}
\newcommand{\Ra}{\Rightarrow}
\newcommand{\la}{\leftarrow}
\newcommand{\ra}{\rightarrow}
\newcommand{\ua}{\uparrow}
\newcommand{\da}{\downarrow}
\def\eop{\vspace{5mm} \hfill $\Box$}
\newtheorem{lem}{Lemma}[section]
\newtheorem{thm}{Theorem}[section]
\newtheorem{cor}{Corollary}[section]
%\newfigure{fig}{Figure}[section]
%\newtable{tab}{Table}[section]
\def\sr{{\cal R}} \def\cs{{\cal C}}
\def\va{{\bf a}} \def\vb{{\bf b}} \def\vc{{\bf c}} \def\vd{{\bf d}}
\def\ve{{\bf e}} \def\vf{{\bf f}} \def\vg{{\bf g}} \def\vh{{\bf h}}
\def\vi{{\bf i}} \def\vj{{\bf j}} \def\vk{{\bf k}} \def\vl{{\bf l}}
\def\vm{{\bf m}} \def\vn{{\bf n}} \def\vo{{\bf o}} \def\vp{{\bf p}}
\def\vq{{\bf q}} \def\vr{{\bf r}} \def\vs{{\bf s}} \def\vt{{\bf t}}
\def\vu{{\bf u}} \def\vv{{\bf v}} \def\vw{{\bf w}} \def\vx{{\bf x}}
\def\vy{{\bf y}} \def\vz{{\bf z}} \def\v0{{\bf 0}}
\def\bigf{\nabla f}
\def\bigt{\nabla}
\def\big2f{\nabla^{2}f}
\def\vlambda{{\bf \lambda}}
\def\QED{\ifhmode\unskip\nobreak\fi\ifmmode\ifinner\else\hskip5pt\fi\fi
  \hbox{\hskip5pt\vrule width5pt height5pt depth1.5pt\hskip1pt}}


\newcommand{\ignore}[1]{}
\def\tuple#1{\langle #1 \rangle}

% hacks in algorithm environment:
%  need to put the tab stop settings on a separate line;
%  also there seems to be extra vertical space, which is
%  removed with a negative vspace

\newenvironment{algorithm}
{\vspace*{-4ex}
\begin{small} 
\begin{sf} \begin{tabbing}
\\X\=xxx\=xxx\=xxx\=xxx\=xxx\=xxx\=xxx\=  \kill}
{\end{tabbing} \end{sf} 
\end{small}
\vspace*{-2ex}}

\hyphenation{prop-a-ga-tion prop-a-gate Ultra-violet tool-kit tool-kits}
%pjs%%%%\input{abb1.tex}
\newcommand{\bs}{\begin{slide}{}}
\newcommand{\es}{\end{slide}}
\newcommand{\be}{\begin{eqnarray*}}
\newcommand{\ee}{\end{eqnarray*}}
\newcommand{\bt}{\begin{tabular}}
\newcommand{\et}{\end{tabular}}
\newcounter{bean}
\newcommand{\bl}[1]{\begin{list}{#1}{\usecounter{bean}}}
\newcommand{\el}{\end{list}}
\newcommand{\bel}[1]{\begin{equation} \label{#1}}
\newcommand{\eel}{\end{equation}}
\newenvironment{mat}{\left[\begin{array}}{\end{array}\right]}
\newenvironment{dt}{\left|\begin{array}}{\end{array}\right|}
\newcommand{\bip}{\langle} \newcommand{\eip}{\rangle}
\newenvironment{arr}{\begin{array}}{\end{array}}
\def\x1{x_{1}} \def\x2{x_{2}} \def\x3{x_{3}}
\newcommand{\ebs}{\end{slide}\begin{slide}{}}
\newcommand{\bc}{\begin{center}}
\newcommand{\ec}{\end{center}}
\newcommand{\Lra}{\Leftrightarrow}
\newcommand{\lra}{\leftrightarrow}
\newcommand{\Llra}{\Longleftrightarrow}
\newcommand{\La}{\Leftarrow}
\newcommand{\Ra}{\Rightarrow}
\newcommand{\la}{\leftarrow}
\newcommand{\ra}{\rightarrow}
\newcommand{\ua}{\uparrow}
\newcommand{\da}{\downarrow}
\def\eop{\vspace{5mm} \hfill $\Box$}
\newtheorem{lem}{Lemma}[section]
\newtheorem{thm}{Theorem}[section]
\newtheorem{cor}{Corollary}[section]
%\newfigure{fig}{Figure}[section]
%\newtable{tab}{Table}[section]
\def\sr{{\cal R}} \def\cs{{\cal C}}
\def\va{{\bf a}} \def\vb{{\bf b}} \def\vc{{\bf c}} \def\vd{{\bf d}}
\def\ve{{\bf e}} \def\vf{{\bf f}} \def\vg{{\bf g}} \def\vh{{\bf h}}
\def\vi{{\bf i}} \def\vj{{\bf j}} \def\vk{{\bf k}} \def\vl{{\bf l}}
\def\vm{{\bf m}} \def\vn{{\bf n}} \def\vo{{\bf o}} \def\vp{{\bf p}}
\def\vq{{\bf q}} \def\vr{{\bf r}} \def\vs{{\bf s}} \def\vt{{\bf t}}
\def\vu{{\bf u}} \def\vv{{\bf v}} \def\vw{{\bf w}} \def\vx{{\bf x}}
\def\vy{{\bf y}} \def\vz{{\bf z}} \def\v0{{\bf 0}}
\def\bigf{\nabla f}
\def\bigt{\nabla}
\def\big2f{\nabla^{2}f}
\def\vlambda{{\bf \lambda}}
\def\QED{\ifhmode\unskip\nobreak\fi\ifmmode\ifinner\else\hskip5pt\fi\fi
  \hbox{\hskip5pt\vrule width5pt height5pt depth1.5pt\hskip1pt}}

\newcommand{\bs}{\begin{slide}{}}
\newcommand{\es}{\end{slide}}
\newcommand{\be}{\begin{eqnarray*}}
\newcommand{\ee}{\end{eqnarray*}}
\newcommand{\bt}{\begin{tabular}}
\newcommand{\et}{\end{tabular}}
\newcounter{bean}
\newcommand{\bl}[1]{\begin{list}{#1}{\usecounter{bean}}}
\newcommand{\el}{\end{list}}
\newcommand{\bel}[1]{\begin{equation} \label{#1}}
\newcommand{\eel}{\end{equation}}
\newenvironment{mat}{\left[\begin{array}}{\end{array}\right]}
\newenvironment{dt}{\left|\begin{array}}{\end{array}\right|}
\newcommand{\bip}{\langle} \newcommand{\eip}{\rangle}
\newenvironment{arr}{\begin{array}}{\end{array}}
\def\x1{x_{1}} \def\x2{x_{2}} \def\x3{x_{3}}
\newcommand{\ebs}{\end{slide}\begin{slide}{}}
\newcommand{\bc}{\begin{center}}
\newcommand{\ec}{\end{center}}
\newcommand{\Lra}{\Leftrightarrow}
\newcommand{\lra}{\leftrightarrow}
\newcommand{\Llra}{\Longleftrightarrow}
\newcommand{\La}{\Leftarrow}
\newcommand{\Ra}{\Rightarrow}
\newcommand{\la}{\leftarrow}
\newcommand{\ra}{\rightarrow}
\newcommand{\ua}{\uparrow}
\newcommand{\da}{\downarrow}
\def\eop{\vspace{5mm} \hfill $\Box$}
\newtheorem{lem}{Lemma}[section]
\newtheorem{thm}{Theorem}[section]
\newtheorem{cor}{Corollary}[section]
%\newfigure{fig}{Figure}[section]
%\newtable{tab}{Table}[section]
\def\sr{{\cal R}} \def\cs{{\cal C}}
\def\va{{\bf a}} \def\vb{{\bf b}} \def\vc{{\bf c}} \def\vd{{\bf d}}
\def\ve{{\bf e}} \def\vf{{\bf f}} \def\vg{{\bf g}} \def\vh{{\bf h}}
\def\vi{{\bf i}} \def\vj{{\bf j}} \def\vk{{\bf k}} \def\vl{{\bf l}}
\def\vm{{\bf m}} \def\vn{{\bf n}} \def\vo{{\bf o}} \def\vp{{\bf p}}
\def\vq{{\bf q}} \def\vr{{\bf r}} \def\vs{{\bf s}} \def\vt{{\bf t}}
\def\vu{{\bf u}} \def\vv{{\bf v}} \def\vw{{\bf w}} \def\vx{{\bf x}}
\def\vy{{\bf y}} \def\vz{{\bf z}} \def\v0{{\bf 0}}
\def\bigf{\nabla f}
\def\bigt{\nabla}
\def\big2f{\nabla^{2}f}
\def\vlambda{{\bf \lambda}}
\def\QED{\ifhmode\unskip\nobreak\fi\ifmmode\ifinner\else\hskip5pt\fi\fi
  \hbox{\hskip5pt\vrule width5pt height5pt depth1.5pt\hskip1pt}}

\newcommand{\bs}{\begin{slide}{}}
\newcommand{\es}{\end{slide}}
\newcommand{\be}{\begin{eqnarray*}}
\newcommand{\ee}{\end{eqnarray*}}
\newcommand{\bt}{\begin{tabular}}
\newcommand{\et}{\end{tabular}}
\newcounter{bean}
\newcommand{\bl}[1]{\begin{list}{#1}{\usecounter{bean}}}
\newcommand{\el}{\end{list}}
\newcommand{\bel}[1]{\begin{equation} \label{#1}}
\newcommand{\eel}{\end{equation}}
\newenvironment{mat}{\left[\begin{array}}{\end{array}\right]}
\newenvironment{dt}{\left|\begin{array}}{\end{array}\right|}
\newcommand{\bip}{\langle} \newcommand{\eip}{\rangle}
\newenvironment{arr}{\begin{array}}{\end{array}}
\def\x1{x_{1}} \def\x2{x_{2}} \def\x3{x_{3}}
\newcommand{\ebs}{\end{slide}\begin{slide}{}}
\newcommand{\bc}{\begin{center}}
\newcommand{\ec}{\end{center}}
\newcommand{\Lra}{\Leftrightarrow}
\newcommand{\lra}{\leftrightarrow}
\newcommand{\Llra}{\Longleftrightarrow}
\newcommand{\La}{\Leftarrow}
\newcommand{\Ra}{\Rightarrow}
\newcommand{\la}{\leftarrow}
\newcommand{\ra}{\rightarrow}
\newcommand{\ua}{\uparrow}
\newcommand{\da}{\downarrow}
\def\eop{\vspace{5mm} \hfill $\Box$}
\newtheorem{lem}{Lemma}[section]
\newtheorem{thm}{Theorem}[section]
\newtheorem{cor}{Corollary}[section]
%\newfigure{fig}{Figure}[section]
%\newtable{tab}{Table}[section]
\def\sr{{\cal R}} \def\cs{{\cal C}}
\def\va{{\bf a}} \def\vb{{\bf b}} \def\vc{{\bf c}} \def\vd{{\bf d}}
\def\ve{{\bf e}} \def\vf{{\bf f}} \def\vg{{\bf g}} \def\vh{{\bf h}}
\def\vi{{\bf i}} \def\vj{{\bf j}} \def\vk{{\bf k}} \def\vl{{\bf l}}
\def\vm{{\bf m}} \def\vn{{\bf n}} \def\vo{{\bf o}} \def\vp{{\bf p}}
\def\vq{{\bf q}} \def\vr{{\bf r}} \def\vs{{\bf s}} \def\vt{{\bf t}}
\def\vu{{\bf u}} \def\vv{{\bf v}} \def\vw{{\bf w}} \def\vx{{\bf x}}
\def\vy{{\bf y}} \def\vz{{\bf z}} \def\v0{{\bf 0}}
\def\bigf{\nabla f}
\def\bigt{\nabla}
\def\big2f{\nabla^{2}f}
\def\vlambda{{\bf \lambda}}
\def\QED{\ifhmode\unskip\nobreak\fi\ifmmode\ifinner\else\hskip5pt\fi\fi
  \hbox{\hskip5pt\vrule width5pt height5pt depth1.5pt\hskip1pt}}


\begin{document}

\title{Solving Linear Arithmetic Constraints \\ 
       for User Interface Applications}

\author{
\parbox[t]{8cm}{\centering
             {\em Alan Borning\thanks{\ \ Permanent address:
Department of Computer Science \& Engineering, University of Washington,
Box 352350, Seattle, WA 98195, USA}\ \ and 
% omitted \mbox{borning@cs.washington.edu} to save a line ...
             Kim Marriott}\\
             Department of Computer Science \\
             Monash University \\
             Clayton, Victoria 3168, AUSTRALIA \\
             \{borning,marriott\}@cs.monash.edu.au} 
\parbox[t]{8cm}{\centering
             {\em Peter Stuckey and Yi Xiao}\\
             Department of Computer Science; \\ 
                  Department of Mathematics \& Statistics \\
             University of Melbourne \\
             Parkville, Victoria 3052, AUSTRALIA  \\
             pjs@cs.mu.oz.au; yxiao@maths.mu.oz.au}
}

\maketitle

\abstract Linear equality and inequality constraints arise naturally in
specifying many aspects of user interfaces, such as requiring that one
window be to the left of another, requiring that a pane occupy the leftmost
1/3 of a window, or preferring that an object be contained within a
rectangle if possible.  Current constraint solvers designed for UI
applications cannot efficiently handle simultaneous linear equations and
inequalities.  This is a major limitation.  We describe incremental
algorithms based on the dual simplex and active set methods
that can solve such systems of constraints
efficiently.  %Both algorithms have been implemented and tested.

\keywords Linear constraints, inequality constraints, simplex algorithm


\section{INTRODUCTION}

Linear equality and inequality constraints arise naturally in specifying
many aspects of user interfaces, in particular layout and other geometric
relations.  Inequality constraints, in particular, are needed to express
relationships such as ``inside,'' ``above,'' ``below,'' ``left-of,''
``right-of,'' and ``overlaps.''  For example, if we are designing a
Web document we can express the
requirement that {\code figure1} be to the left of {\code figure2} as the
constraint \mbox{\code figure1.rightSide $\leq$ figure2.leftSide}.

It is important to be able to express preferences as well as requirements
in a constraint system.  One use is to express a desire for stability
when moving parts of an image: things should stay where they were unless
there is some reason for them to move.  A second use is to process
potentially invalid user inputs in a graceful way.  For example, if the
user tries to move a figure outside of its bounding window, it is
reasonable for the figure just to bump up against the side of the window
and stop, rather than giving an error.  A
third use is to balance conflicting desires, for example in laying out a
graph.

Efficient techniques are available for solving such constraints if the
constraint network is acyclic.  However, in trying to apply constraint
solvers to real-world problems, we found that the collection of constraints
was in fact often cyclic.  This sometimes arose when the programmer
added redundant constraints --- the cycles {\em could} have been avoided by
careful analysis.  However, this is an added burden on the programmer.
Further, it is clearly contrary to the spirit of the whole enterprise to
require programmers to be constantly on guard to avoid cycles and redundant
constraints; after all, one of the goals in providing constraints is to
allow programmers to state what relations they want to hold in a
declarative fashion, leaving it to the underlying system to enforce these
relations.  For other applications, such as complex layout problems with
conflicting goals,  cycles seem unavoidable.

\subsection{Constraint Hierarchies and Comparators}

Since we want to be able to express preferences as well as requirements in
the constraint system, we need a specification for how conflicting
preferences are to be traded off.  {\em Constraint hierarchies}
\cite{borning-lisp-symbolic-computation-92} provide a general theory for
this.  In a constraint hierarchy each constraint has a strength.  The
{\strength required} strength is special, in that {\strength required}
constraints must be satisfied.  The other strengths all label non-required
constraints.  A constraint of a given strength completely dominates any
constraint with a weaker strength.  In the theory, a {\em comparator} is
used to compare different possible solutions to the constraints and select
among them.

\ignore{
Within this framework a number of variations are possible.  One choice is
whether we only compare solutions on a constraint-by-constraint basis (a
{\em local} comparator), or whether we take some aggregate measure of the
unsatisfied constraints of a given strength (a {\em global} comparator).  A
second choice is whether we are concerned only whether a constraint is
satisfied or not (a {\em predicate} comparator), or whether we also want to
know how nearly satisfied it is (a {\em metric} comparator.  (Constraints
whose domain is a metric space, for example the reals, can have an
associated error function.  The error in satisfying a constraint {\em cn}
is 0 iff the constraint is satisfied, and becomes larger the less nearly
satisfied is the constraint.)
}

As described in~\cite{borning-uist-96},
 it is important to use a metric rather than a
predicate comparator for inequality constraints.  Thus, plausible comparators for use with linear
equality and inequality constraints are {\em locally-error-better}, 
{\em weighted-sum-better}, and {\em least-squares-better}.  
\ignore{For a given
collection of constraints, Cassowary finds a locally-error-better or a
weighted-sum-better solution; QOCA finds a least-squares-better solution.}
The least-squares-better comparator strongly
penalizes outlying values when trading off constraints of the same
strength.  It is particularly suited to tasks such as laying out a tree, a
graph, or a collection of windows, where there are inherently conflicting
preferences (for example, that all the nodes in the depiction of a graph
have some minimum spacing and that edge lengths be minimized).
Locally-error-better, on the other hand, is a more permissive
comparator, in that it admits more solutions to the constraints.  (In fact
any least-squares-better or weighted-sum-better solution is also a
locally-error-better solution \cite{borning-lisp-symbolic-computation-92}.)
It is thus easier to implement algorithms to find a locally-error-better
\linebreak
solution, and in particular to design hybrid algorithms that include
subsolvers for simultaneous equations and inequalities and also subsolvers
for nonnumeric constraints \cite{borning-cp-95}.
Since each of these different comparators is preferable in certain
situations we  give algorithms for each.

% Also our implementation of Cassowary favors solutions that satisfies
% some of the constraints completely -- this may be more pleasing to
% users; but probably we don't want to say this here.  Also this needs to
% be tested empirically. 

\subsection{Adapting the Simplex Algorithm}

Linear programming is concerned with solving the following problem.  Consider
a collection of $n$ real-valued variables $x_1, \ldots, x_n$, each
of which  is constrained to be non-negative: 
$x_i \geq 0$ for $1 \leq i \leq n$.  There are  $m$
linear equality or inequality constraints over the $x_i$, each of the form
\hspace*{5mm}\mbox{$a_1 x_1 + \ldots + a_n x_n = b$},\\
\hspace*{5mm}\mbox{$a_1 x_1 + \ldots + a_n x_n \leq b$},  or\\
\hspace*{5mm}\mbox{$a_1 x_1 + \ldots + a_n x_n \geq b$}.\\
Given these constraints, we wish to find values for the $x_i$ that minimizes 
(or maximizes) the value of the {\em objective function}
\hspace*{5mm}$c + d_1 x_1 + \ldots + d_n x_n$. \\
This problem has been heavily studied for the past 50 years.  The most
commonly used algorithm for solving it is the simplex algorithm, developed
by Dantzig in the 1940s, and there are now numerous variations of it. 
Unfortunately, however, existing implementations of the simplex
are not really suitable for UI applications.

The principal issue is incrementality.  In UI
applications, we need to solve similar problems repeatedly, rather than
solving a single problem once, and to achieve interactive response times,
very fast incremental algorithms are needed.  There are two cases.  First, when
moving an object with a mouse or other input device, we typically represent
this interaction as a one-way constraint relating the mouse position to the
desired $x$ and $y$ coordinates of a part of the figure.  For this case we
must resatisfy the same collection of constraints, differing only in the
mouse location, each time the screen is refreshed.  Second, when editing an
object we may add or remove constraints and other parts, and we would like
to make these operations fast, by reusing as much of the previous solution
as possible.  The performance requirements are considerably more stringent
for the first case than the second.  
\ignore{In Sections \ref{resolving} and
\ref{quadratic} we describe how to update an existing solution rapidly
given new inputs (e.g.\ a new mouse position), while in Sections
\ref{adding-constraints} and \ref{removing-constraints} we describe how to
add or delete a constraint incrementally.}

Another issue is defining a suitable objective function.  The objective
function in the standard simplex algorithm must be a linear expression; but
the objective functions for the locally-error-better,
weighted-sum-better, and least-squares-better comparators are all
non-linear.  Fortunately techniques have been developed in the operations
research community for handling these cases, which we adopt here.  For the
first two comparators, the objective functions are ``almost linear,''
while the third comparator gives rise to a quadratic
optimization problem.
\ignore{leading to the quasi-linear optimization technique described in Section
\ref{quasi-linear}.  Least-squares-better results in a quadratic
optimization problem, which is solved using the technique described in
Section \ref{quadratic}. } 

Finally, a third issue is accommodating variables that may take on both
positive and negative values, which in general is the case in UI
applications.  (The standard simplex algorithm requires all variables to be
non-negative.)  Here we adopt efficient techniques developed for
implementing constraint logic programming languages.

\subsection{Overview}

We present algorithms for incrementally solving linear equality and
inequality constraints for the three different comparators described
above. In Section~\ref{augmented-simplex-form} we give algorithms for
incrementally adding and deleting required constraints with restricted and
unrestricted variables from a system of constraints kept in {\em augmented
simplex form}, a type of solved form.  In Section \ref{resolving} we give
an algorithm, Cassowary, based on the dual simplex, for incrementally
solving hierarchies of constraints using the locally-error-better or
weighted-sum-better comparators when a constraint is added or an object is
moved, while in Section \ref{quadratic} we give an algorithm, QOCA, based
on the active set method, for incrementally solving hierarchies of
constraints using the least-squares-better comparator.

Both of our algorithms have been implemented,
Cassowary in Smalltalk and QOCA in C++\@. 
They perform surprisingly well, and
a summary of our results is given in Section \ref{empirical-evaluation}. 
The QOCA
implementation is considerably more sophisticated and has much better
performance than the current version of Cassowary.  
However, QOCA is inherently a more
complex algorithm, and re-implementing it with a comparable level of
performance would be a daunting task.    
In contrast, Cassowary
is straightforward, and a reimplementation based on this paper is more
reasonable, given a knowledge of the simplex algorithm.  A companion
technical report \cite{borning-simplex-tr} gives additional details for
both algorithms.

\subsection{Related Work}

There is a long history of using constraints in user interfaces and
interactive systems, beginning with Ivan Sutherland's pioneering Sketchpad
system \cite{sutherland-ifips-63}.  Most of the current systems use one-way
constraints (e.g.\ \cite{hudson-subarctic-manual,myers-chi-96}), or local
propagation algorithms for acyclic collections of multi-way constraints
(e.g.\ \cite{sannella-spe-93,vander-zanden-toplas-96}).
Indigo \cite{borning-uist-96} handles acyclic collections of inequality
constraints, but not cycles (simultaneous equations and inequalities).  UI
systems that handle simultaneous linear equations include \mbox{DETAIL}
\cite{hosobe-cp-96} and Ultraviolet \cite{borning-cp-95}.  A number of
researchers (including the first author) have experimented with a
straightforward use of a simplex package in a UI constraint solver,
but the speed was not satisfactory for interactive use.
An earlier version of QOCA is described in references \cite{helm-gi-92} and
\cite{helm-eurographics-92}.  These earlier descriptions, however, do not
give any details of the algorithm, although the incremental deletion
algorithm is described in \cite{huynh-marriott-96}.  
The current implementation is
much improved, in particular through the use of the active set method
described in Section \ref{active-sets}.

Baraff \cite{baraff-siggraph-94} describes a quadratic optimization
algorithm for solving linear constraints that arise in modelling physical
systems.  Finally, much of the work on constraint solvers
has been in the logic programming and constraint logic programming
communities.  Current constraint logic programming languages such as
CLP($\cal R$) \cite{jaffar-toplas-92} include efficient solvers for linear
equalities and inequalities.  (See \cite{marriott-stuckey-book} for a
survey.)  However, these solvers use a refinement model of computation, in
which the values determined for variables are successively refined as the
computation progresses, but there is no notion as such of state and change.
As a result, these systems are not so well suited for building interactive
graphical applications.

% The language Bertrand \cite{leler-book} also
% solves simultaneous linear constraints (but not inequalities).

\pagebreak

\section{INCREMENTAL SIMPLEX}
\label{inc-simplex}

\begin{quotation}
As you see, the subject of linear programming is surrounded by notational
and terminological thickets.  Both of these thorny defenses are lovingly
cultivated by a coterie of stern acolytes who have devoted themselves to
the field.  
Actually, the basic ideas of linear programming are quite simple. 
-- {\em Numerical Recipes}, \cite[page 424]{press-89}
\end{quotation}

We now describe an incremental version of the simplex algorithm, adapted to
the task at hand.
%The material presented in this section is common to both
%Cassowary and QOCA\@.  
%The two algorithms use different optimization
%techniques, however, which are described in Sections \ref{quasi-linear} and
%\ref{quadratic} respectively.  
In the description we use a running example,
illustrated by the diagram in Figure~\ref{fig:pict}.

\begin{figure}[htb]
\begin{center}
\input constraints.eepic
\end{center}
\caption{Simple constrained picture\label{fig:pict}}
\end{figure}

The constraints on the variables in Figure~\ref{fig:pict} are as follows:
$x_m$ is constrained to be the midpoint of the line from $x_l$ to $x_r$,
and $x_l$ is constrained to be at least 10 to the left of $x_r$.  All
variables must lie in the range 0 to 100.  
%(To keep the presentation
%manageable, we deal only with the $x$ coordinates.  Adding analogous
%constraints on the $y$ coordinates would be simple but would double the
%number of the constraints in our example.)  
Since $x_l < x_m < x_r$ in any
solution, we simplify the problem by removing the redundant bounds
constraints.  However,
even with these simplifications the resulting constraints have a cyclic
constraint graph, and cannot be handled by methods such as Indigo.

We can represent this using the constraints
$$\begin{array}{rcl}
2 x_m &=& x_l + x_r \\
x_l + 10 &\leq &x_r \\
x_r &\leq& 100 \\
0 &\leq& x_l
\end{array}$$
Now suppose  we wish to minimize the distance between
$x_m$ and $x_l$ or in other words, minimize $x_m - x_l$.

\subsection{Augmented Simplex Form}
\label{augmented-simplex-form}

An optimization problem is in \emph{augmented simplex form} if constraint
$C$ has the form $C_U \wedge C_S \wedge C_I$, where $C_U$ and $C_S$ are
conjunctions of linear arithmetic equations and $C_I$ is 
\mbox{$\bigwedge \{ x
\geq 0 \mid \mbox{$x$ is a variable in $C_S$}\}$}, and 
the objective function $f$ is a linear
expression over variables in $C_S$\@.  The simplex algorithm does not itself
handle variables that may take negative values (so-called {\em
unrestricted variables}), and imposes a constraint $x \geq 0$ on all
variables occurring in its equations.  Augmented simplex form allows us to
handle unrestricted variables efficiently and simply; it was developed for
implementing constraint logic programming languages
\cite{marriott-stuckey-book}, and we have adopted it here.  Essentially it
uses {\em two} tableaux rather than one.  
Equations containing at least one unrestricted
variable will be placed in $C_U$, the unrestricted variable tableau while
$C_S$, the simplex tableau, contains those equations in 
which all variables are constrained to be
non-negative.  The simplex algorithm is used to determine an optimal
solution for the equations in the simplex tableau, ignoring the
unrestricted variable tableau for purposes of optimization.  The equations
in the unrestricted variable tableau are then used to determine values for
its variables.

It is not difficult to write an arbitrary optimization problem over linear
real equations and inequalities into augmented simplex form.  The first
step is to convert inequalities to equations.  Each inequality of the form
$e \leq r$, where $e$ is a linear real expression and $r$ is a number, can be
replaced with $e + s = r \wedge s \geq 0$ where $s$ is a new non-negative
\emph{slack} variable.

For example, the constraints for Figure~\ref{fig:pict} can be written as
\begin{quote}\vspace*{-1ex}
minimize $x_m - x_l$ 
subject to 
$$\begin{array}{rcl}
2 x_m & = & x_l + x_r \\
x_l + 10 + s_1& = &x_r \\
x_r + s_2 &= &100 \\
0 &\leq & x_l, s_1, s_2
\end{array}$$
\end{quote}\vspace{-0.9ex}

We now separate the equalities into $C_U$ and $C_S$\@.
Initially all equations are in $C_S$\@.  We separate out the
unrestricted variables into $C_U$ using Gauss-Jordan elimination.  To do
this, we select an equation in $C_S$ containing an unrestricted variable
$u$ and remove the equation
from $C_S$\@.  We then solve the equation for $u$, yielding
a new equation $u=e$ for some expression $e$\@.  We then substitute $e$ for
all remaining occurrences of $u$ in $C_S$, $C_U$, and $f$,
and place the equation $u=e$ in
$C_U$\@.  The process is repeated until there are no more unrestricted
variables in $C_S$\@.  In our example the third equation can be used to
substitute $100 - s_2$ for $x_r$ 
%obtaining
%\begin{quote}\vspace*{-1ex}
%minimize $x_m - x_l$ 
%$$
%\begin{array}{rcl}
%x_r &= & 100 - s_2 \\ \hline
%2 x_m & = & x_l + 100 - s_2 \\
%x_l + 10 + s_1& = & 100 - s_2 \\
%0 &\leq & x_l, s_1, s_2
%\end{array}
%$$
%\end{quote}\vspace{-0.9ex}
%Next, 
and the first equation can be used to 
substitute $50 + \frac{1}{2}x_l - \frac{1}{2} s_2$ 
for $x_m$, giving 
\begin{quote}\vspace*{-1ex}
minimize $50 - \frac{1}{2}x_l - \frac{1}{2} s_2$ 
subject to 
$$
\begin{array}{rcl}
x_m & = & 50 + \frac{1}{2} x_l - \frac{1}{2} s_2 \\ 
x_r &= & 100 - s_2 \\ \hline
x_l + 10 + s_1& = &100 - s_2 \\
0 &\leq & x_l, s_1, s_2
\end{array}
$$
\end{quote}\vspace{-0.9ex}
The tableau shows $C_U$ above the horizontal line,
and $C_S$ and $C_I$ below the horizontal line.  From now 
on $C_I$ will be omitted --- any variable occurring below the horizontal
line is implicitly constrained to be non-negative.
The simplex method works by taking a an optimization problem in ``basic
feasible solved form'' (a type of normal form) and repeatedly applying
matrix operations to obtain new basic feasible solved forms.  Once we
have split the equations into $C_U$ and $C_S$ we can ignore $C_U$ for
purposes of optimization. 

% The following isn't completely accurate, since we need to do something
% special for updating the stays if we don't do the pivots everywhere:
% (so I'm just leaving it out):
%
% We illustrate the operations as modifying $C_U$
% so as to show the solutions more easily, but in practice $C_U$ remains
% unchanged and we use it only to define values for the variables on the left
% hand side of its equations.

A augmented simplex form optimization problem is in 
\emph{basic feasible solved form} if the equations are of the form 
$$x_0 = c + a_1 x_1 + \ldots + a_n x_n$$ where the variable $x_0$ does not
occur in any other equation or in the objective function.  If the equation
is in $C_S$, $c$ must be non-negative.  However, there is no
such restriction on the constants for the equations in $C_U$\@.  In either
case the variable $x_0$ is said to be \emph{basic} and the other
variables in the equation are \emph{parameters}.  A problem in basic
feasible solved form defines a \emph{basic feasible solution}, which is
obtained by setting each parametric variable to 0 and each basic variable
to the value of the constant in the right-hand side.

For instance, the following constraint
is in basic feasible solved form and is equivalent to the 
problem above.
\begin{quote}\vspace*{-1ex}
minimize $50 - \frac{1}{2} x_l + \frac{1}{2} s_2 $ 
subject to 
$$
\begin{array}{rlrrr} 
x_m & = &50 & + \frac{1}{2} x_l & - \frac{1}{2} s_2 \\
x_r & = &100 &  & - s_2 \\ \hline
s_1 & = &90 & - x_l &  - s_2 
\end{array}
$$
\end{quote}\vspace{-0.9ex}
The basic feasible solution corresponding to this
basic feasible solved form is 
$$\{x_m \mapsto 50, x_r \mapsto 100, s_1 \mapsto 90, x_l \mapsto 0, 
s_2 \mapsto 0\}.$$
The value of the objective function with this solution is 50.

\subsection{Simplex Optimization}
\label{simplex-optimization}

We now describe how to find an optimum solution to a constraint in basic
feasible solved form.  Except for the operations on the additional
unrestricted variable tableau $C_U$, the material presented in this
subsection is simply the second phase of the standard two-phase simplex algorithm.

The simplex algorithm finds the optimum by repeatedly looking for an
``adjacent'' basic feasible solved form whose basic feasible solution
decreases the value of the objective function.  When no such adjacent basic
feasible solved form can be found, the optimum has been found.  The
underlying operation is called {\em pivoting}, and involves exchanging a
basic and a parametric variable using matrix operations.  Thus by
``adjacent'' we mean the new basic feasible solved form can be reached by
performing a single pivot.

In our example, increasing $x_l$ from $0$ will decrease the value of the
objective function.  We must be careful as we cannot increase the value of
$x_l$ indefinitely as this may cause the value of some other
basic non-negative variable to become negative.  We must examine the
equations in $C_S$\@.  The equation $s_1 = 90 - x_l - s_2$ allows $x_l$ to
take at most a value of $90$, as if $x_l$ becomes larger than this, then
$s_1$ would become negative.  The equations above the horizontal line do not
restrict $x_l$, since whatever value $x_l$ takes the unrestricted variables
$x_m$ and $x_r$ can take a value to satisfy the equation.  In general,
we choose the most restrictive equation in $C_S$, and use it to eliminate
$x_l$\@.  In the case of ties we arbitrarily break the tie. In this example
the most restrictive equation is $s_1 = 90 - x_l - s_2$\@.  Writing $x_l$ as
the subject we obtain $x_l = 90 - s_1 - s_2$\@.  We replace $x_l$ everywhere
by $90 - s_1 - s_2$ and obtain
\begin{quote}\vspace*{-1ex}
minimize $5 + \frac{1}{2} s_1 + s_2$ 
subject to 
$$
\begin{array}{rlrrr} 
x_m & = &95 & - \frac{1}{2} s_1 & - s_2 \\
x_r & = &100 &  & - s_2 \\ \hline
x_l & = &90 & - s_1 & - s_2 
\end{array}
$$
\end{quote}\vspace{-0.9ex}

We have just performed a pivot, having moved $s_1$ out of the set of basic
variables and replaced it by $x_l$\@.

We continue this process.  Increasing the value of $s_1$ will increase the
value of the objective.  Note that decreasing $s_1$ will also decrease the
objective function value, but as $s_1$ is constrained to be non-negative,
it already takes its minimum value of $0$ in the associated basic feasible
solution.  Hence we are at an optimal solution.  

%(If we were to have an
%unrestricted variable in the objective function, the optimization would be
%unbounded.  This is not an issue for Cassowary or QOCA, since the objective
%function in those cases always only contains non-negative variables.)

% We could change the name ``non-negative variable'' to ``slack variable''
% if we always made user variables be unrestricted variables.  This would
% require making x_l an unrestricted variable in the example.

In general, the simplex algorithm applied to $C_S$ 
is described as follows. 
We are given a problem in basic feasible solved form in which 
the variables $x_1, \ldots ,x_n$ are basic and the variables
$y_1, \ldots ,y_m$ are parameters.
\begin{quote}\vspace*{-1ex}
minimize $e + \sum_{j=1}^{m} d_j y_j$
subject to 
$$
\begin{array}{rcr}
        \bigwedge_{i=1}^{n} x_i & = & c_i + \sum_{j=1}^{m} a_{ij} y_j ~\wedge\\
                  \multicolumn{3}{c}{\bigwedge_{i=1}^{n} x_i \ge 0 \wedge
		  \bigwedge_{j=1}^{m} y_j \ge 0. }
            \end{array}
$$
\end{quote}\vspace{-0.9ex}
Select an entry variable $y_J$ such that $d_J < 0$\@.  (An entry variable is
one that will enter the basis, i.e.\ it is currently parametric and we want
to make it basic.)
Pivoting on such a variable can only decrease the value of the objective
function.
If no such variable exists, the optimum has been reached.
Now determine the exit variable $x_I$\@.  We must choose this variable so that
it maintains  basic feasible solved form by ensuring 
that the new $c_i$'s are still positive after pivoting. 
This is achieved by choosing an $x_I$ 
so that $- c_I/ a_{IJ}$ is a minimum element
of the set
$$
\{ -c_i/a_{iJ} \mid a_{iJ} < 0 \mbox{ and } 1 \le i \le n \}.
$$
If there were no $i$ for which $a_{iJ} < 0$ then we could stop since the
optimization problem would be unbounded, and so would not have
a minimum.  This is not an issue in our context since our
optimization problems will always have a lower bound of 0.
% This is because we can choose $y_J$ to take an arbitrarily
% large value, and so make the objective function arbitrarily small.
We proceed to choose $x_I$, and pivot $x_I$ out and replace it with $y_J$
to obtain the new basic feasible solution.
We continue this process until an optimum is reached.
%The algorithm is illustrated in Figure~\ref{fig:simplex-opt},
%and takes as inputs the simplex tableau $C_S$ and the objective function $f$\@.

\ignore{
\begin{figure}[tb]
\begin{center}
\fbox{
\begin{minipage}{10cm}
\begin{tabbing}
xx \= xx \= xx \= xx \= \kill
%Boolean
\textsf{simplex\_opt}($C_S$,$f$) \\
\> \textbf{repeat} \\
\> \> \% Choose variable $y_J$ to become basic\\
\> \> \textbf{if} for each $j \in \{ 1, \ldots ,m \}$ $d_j \ge 0$ \textbf{then} \\
\> \> \> \textbf{return} \% an optimal solution has been found\\
\> \> \textbf{endif} \\
\> \> \textbf{choose} $J \in \{ 1, \ldots ,m \}$ such that $d_J < 0$\\
\> \> \% Choose variable $x_I$ to become non-basic\\
\> \> \textbf{choose} $I \in \{ 1, \ldots ,n \}$ such that\\
\> \> \> $-c_I / a_{IJ} =
    \min_{i \in \{1,\ldots, n\}} \{ -c_i / a_{iJ} \mid a_{iJ} < 0 \}$\\
\> \> $e$ := $( x_I - c_I - \sum_{j=1, j\ne J}^m a_{Ij} y_j)/ a_{IJ}$\\
\> \> $C_S \left[ I \right]$ := $(Y_J = e)$\\
\> \> replace $Y_J$ by $e$ in $f$\\
\> \> \textbf{for} each $i \in \{1, \ldots ,n\}$\\
\> \> \> \textbf{if} $i \ne I$ \textbf{then}
       replace $Y_J$ by $e$ in $C_S \left[ I \right]$ \textbf{endif}\\
\> \> \textbf{endfor}\\
\> \textbf{endrepeat}
\end{tabbing}
\end{minipage}
}
\end{center}
\caption{Simplex optimization\label{fig:simplex-opt}}
\end{figure}
}

\ignore{ %%%%%%%%%%%%%%%%%%%%%%%%%%%%%%%%%%%%%%%%%%%%%%%%%
\begin{quote}\vspace*{-1ex}
\textbf{Simplex algorithm} 
\\ {\sc input:} An  optimization problem $(C_S \wedge C_P,f)$ 
in basic feasible solved form.
\\ {\sc output:} Either  $false$ indicating that $(C_S \wedge C_P,f)$
does not have an optimal solution or else an optimal solution
to  $(C_S \wedge C_P,f)$.
\\ {\sc method:} 
Call , and let $\tuple{ F, C', f'}$ be the result.
If $F$ is $false$, output $false$, otherwise output the basic feasible
solution corresponding to $(C',f')$.
\end{quote}\vspace{-0.9ex}
} %%%%%%%%%%%%%%%%%%%%%%%%%%%%%%%%%%%%%%%%%%%%%%%%%%%

\subsection{Incrementality: Adding a Constraint}
\label{adding-constraints}

We now describe how to add the equation for a new constraint incrementally.
This technique is also used in our implementations to find an initial basic
feasible solved form for the original simplex problem, by starting from an
empty constraint set and adding the constraints one at a time.

As an example, suppose we wish to ensure the additional constraint
that the midpoint
sits in the centre of the screen.  
This is represented by the constraint
$x_m = 50$.  If we substitute for each of the basic variables
(only $x_m$) in this constraint we obtain the equation
$45 - \frac{1}{2} s_1 - s_2 = 0$.  
In order to add this constraint
straightforwardly to the tableau we create a new 
non-negative variable $a$ called an \emph{artificial variable}.
(This is simply an incremental version of the operation used in
the first phase of the two-phase simplex algorithm.)
We let $a = 45 - \frac{1}{2} s_1 - s_2$ be added to the tableau
(clearly this gives a tableau in basic feasible solved form)
and then minimize the value of $a$.
If $a$ takes the value $0$
then we have obtained a solution to the problem
with the added constraint, and
we can then eliminate the artificial variable altogether since it is 
a parameter (and hence takes the value 0).  This is the case for our example;
the resulting tableau is
\begin{quote}\vspace*{-1ex}
$$
\begin{array}{rlrrr} 
x_m & = &50   \\
x_r & = &100 & - s_2 \\ \hline
x_l & = &0 & + s_2 \\
s_1 & = &90 & -2 s_2
\end{array}
$$
\end{quote}\vspace{-0.9ex}

In general, to add a new required constraint to the tableau we first convert it
to an augmented simplex form equation by adding slack variables if it is an
inequality.  Next, we use the current tableau to substitute out all the
basic variables.  This gives an equation $e = c$ where $e$ is a linear
expression.  If $c$ is negative, we multiply both sides by $-1$ so that the
constant becomes non-negative.  If $e$ contains an unrestricted variable
we use it to substitute for that variable and add the equation to the
tableau above the line (i.e.\ to $C_U$)\@.  Otherwise we create an
non-negative artificial variable $a$ and add the equation $a = c - e$ to
the tableau below the line (i.e.\ to $C_S$), and minimize $c - e$. If the
resulting minimum is not zero then the constraints are unsatisfiable.
Otherwise $a$ is either parametric or basic.  If $a$ is parametric, the
column for it can be simply removed from the tableau.  If it is basic, the
row must have constant 0 (since we were able to achieve a value of 0 for
our objective function, which is equal to $a$).  If the row is just 
$a = 0$, it can be removed.  Otherwise, $a = 0 + b x + e$ where $b \neq 0$.
We can then pivot $x$ into the basis using this row and remove the column
for $a$.

% adding a non-required constraint -- particularly efficient -- 
% case of adding an edit 

\subsection{Incrementality: Removing a Constraint}
\label{removing-constraints}

We also want a method for incrementally removing a constraint from the
tableaux.  After a series of pivots have been performed, the information
represented by the constraint may not be contained in a single row, so we
need a way to identify the constraint's influence in the tableaux.  To do
this, we use a ``marker'' variable that is originally present only in the
equation representing the constraint.  We can then identify the
constraint's influence in the tableaux by looking for occurrences of that
marker variable.  For inequality constraints, the slack variable $s$ added
to make it an equality serves as the marker, since $s$ will originally
occur only in that equation.  The equation representing a nonrequired
equality constraint will have an error variable that can serve as a marker
--- see Section \ref{non-requireds}.  For required equality constraints, we
add a ``dummy'' nonnegative variable to the original equation to serve as a
marker, which we never allow to enter the basis (so that it always has
value 0).  In our running example, then, to allow the constraint $2 x_m =
x_l + x_r$ to be deleted incrementally we would add a dummy variable $s_3$,
resulting in $2 x_m = x_l + x_r + s_3$.  The simplex optimization routine
checks for these dummy variables in choosing an entry variable, and does
not allow one to be selected.  (For simplicity we didn't include this
variable in the tableaux presented earlier.)

Consider removing the constraint that $x_l$ is 10 to the left of $x_r$.
The slack variable $s_1$, which we added to the inequality to make it
an equation, records exactly how this equation has been used to modify the
tableau.  We can remove the inequality by pivoting the tableau until 
$s_1$ is basic and then simply drop the row in which it is basic.

In the tableau above $s_1$ is already basic, and so removing it simply means
dropping the row in which it is basic, obtaining
\begin{quote}\vspace*{-1ex}
$$
\begin{array}{rlrrr} 
x_m & = &50   \\
x_r & = &100 & - s_2 \\ \hline
x_l & = &0 & + s_2 \\

\end{array}
$$
\end{quote}\vspace{-0.9ex}

If we wanted to remove this constraint
from the tableau before adding $x_m = 50$ (i.e. the final tableau given in
Section \ref{simplex-optimization}), $s_1$ is a parameter.
We make $s_1$ basic by treating it as an entry variable,
determining the most restrictive equation, and using that equation
to pivot $s_1$ into the basis.
(See \cite{borning-simplex-tr} for details.)
We then remove the row.  Here
the row $x_l  = 90  - s_1  - s_2$
is the most constraining equation. 
Pivoting to let $s_1$ enter the
basis, and then removing the row in which it is basic, we obtain
\begin{quote}\vspace*{-1ex}
$$
\begin{array}{rlrrr} 
x_m & = &50 & + \frac{1}{2} x_l & - \frac{1}{2} s_2 \\
x_r & = &100 &  & - s_2 \\ \hline
\end{array}
$$
\end{quote}\vspace{-0.9ex}

\subsection{Handling Non-Required Constraints}
\label{non-requireds}

Suppose the user wishes to edit $x_m$ in the diagram and have $x_l$ and
$x_r$ weakly stay where they are.  This adds the non-required constraints
$x_m$ {\em edit}, $x_l$ {\em stay}, and $x_r$ {\em stay}.  Suppose further
that we are trying to move $x_m$ to position 50, and that $x_l$ and $x_r$ are
currently at 30 and 60 respectively.  We are thus imposing the constraints
{\strength strong} $x_m = 50$, {\strength weak} $x_l = 30$, and 
{\strength weak} $x_r = 60$.
There are two possible translations of these non-required constraints
to an objective function, depending on the comparator used.

For locally-error-better or weighted-sum-better, we can
simply add the errors of the each constraint to form an objective function.
Consider the constraint $x_m = 50$.  We define the error as $|x_m-50|$\@.  We
need to combine the errors for each non-required constraint with a weight
so we obtain the objective function 
\mbox{$s |x_m - 50| + w |x_l - 30| + w |x_r - 60|$}, where 
$s$ and $w$ are weights so that the strong constraint is always
strictly more important than solving any combination of weak constraints,
so that we find a locally-error-better or weighted-sum-better solution.
For the least-squares-better comparator the objective function is 
$s (x_m - 50)^2 + w (x_l - 30)^2 + w (x_r - 60)^2$\@.  
In the presentation, we will use $s = 1000$ and $w = 1$.
(Cassowary actually uses symbolic weights
and a lexicographic ordering, which ensures that
strong constraints are always satisfied in preference to weak 
ones \cite{borning-simplex-tr}.  However,
QOCA is not able to employ symbolic weights.)

% The optimization function we use for this is not able to handle 
% lexicographic ordering and
% so we will choose $s = 100000$ and $w = 1$. 
% AB: it looks like actually s=1000 is used later

Unfortunately neither of these objective functions is linear and hence the
simplex method is not applicable directly.  We now show how we can 
solve the problem using optimization algorithms
based on the two alternate objective functions: \emph{quasi-linear
optimization} and \emph{quadratic optimization}.

\section{CASSOWARY: QUASI-LINEAR OPTIMIZATION}
\label{quasi-linear}

Cassowary finds either locally-error-better or weighted-sum-better
solutions.  Since every weighted-sum-better solution is also a
locally-error-better solution \cite{borning-lisp-symbolic-computation-92};
the weighted-sum part of the optimization comes automatically from the
manner in which the objective function is constructed.

For Cassowary
both the edit and the stay constraints will be represented as equations of
the form $v = \alpha + \delta_v^{+} - \delta_v^{-}$, 
where $\delta_v^{+}$ and $\delta_v^{-}$ are non-negative variables
representing the
deviation of $v$ from the desired value $\alpha$.  If the constraint is
satisfied both $\delta_v^{+}$ and $\delta_v^{-}$ will be 0.  
Otherwise $\delta_v^{+}$ will be
positive and $\delta_v^{-}$ will be 0 if $v$ is too big, 
or vice versa if $v$ is
too small.  
%(We need to use the pair of variables to satisfy simplex's
%non-negativity restriction.)  
Since we want $\delta_v^{+}$ and $\delta_v^{-}$ to be 0 if
possible, we make them part of the objective function, with larger
coefficients for the error variables for stronger constraints.

Translating the constraints
{\strength strong} $x_m = 50$, 
{\strength weak} $x_l = 30$,
and {\strength weak} $x_r = 60$
that arise from the 
edit and stay constraints we obtain:
$$\begin{array}{rcl}
x_m & = &50 + \delta_{x_m}^+ -  \delta_{x_m}^- \\
x_l & = &30 + \delta_{x_l}^+ -  \delta_{x_l}^- \\
x_r &= &60 + \delta_{x_r}^+ -  \delta_{x_r}^- \\
0 &\leq& \delta_{x_m}^+, \delta_{x_m}^-, \delta_{x_l}^+, \delta_{x_l}^-, 
	\delta_{x_r}^+, \delta_{x_r}^-
\end{array}$$
The objective function to satisfy the
non-required constraints 
can now be restated as \\
\hspace*{5mm}
minimize $1000 \delta_{x_m}^+ + 1000  \delta_{x_m}^- + \delta_{x_l}^+ +  
	\delta_{x_l}^- + \delta_{x_r}^+ +  \delta_{x_r}^-$


\ignore{ %%%%%%%%%%%%%%%%%%%%%%%%%%%%%%%%%%%%%%%%%%%%%%%%%%%%%%
For example the constraints above representing the
constrained picture
can be written in augmented simplex form as
\begin{quote}\vspace*{-1ex}
minimize $1000 \delta_{x_m}^+ + 1000  \delta_{x_m}^- + \delta_{x_l}^+ +  
	\delta_{x_l}^- + \delta_{x_r}^+ +  \delta_{x_r}^-$
subject to 
$$\begin{array}{rcl}
2 x_m = x_l + x_r \\
x_r + s_2 = 100 \\ \hline
x_l + 10 + s_1 = x_r \\
x_m = 50 + \delta_{x_m}^+ -  \delta_{x_m}^- \\
x_l = 30 + \delta_{x_l}^+ -  \delta_{x_l}^- \\
x_r = 90 + \delta_{x_r}^+ -  \delta_{x_r}^- \\
0 \leq x_l, x_m, x_r, s_1, s_2, s_3, s_4, \delta_{x_m}^+, \delta_{x_m}^-, \delta_{x_l}^+, \delta_{x_l}^-, 
	\delta_{x_r}^+, \delta_{x_r}^-
\end{array}$$
\end{quote}\vspace{-0.9ex}

For instance, the following constraint
is in basic feasible solved form and is equivalent to the 
problem above.
\begin{trivlist}\item
minimize $40 - 501 s_2 + 499 x_l + 2000 \delta_{x_m}^- + 2  \delta_{x_l}^+ +
2\delta_{x_r}^- $ 
subject to 
$$
\begin{array}{rlrrrrrr} 
x_m & = &50 & + \frac{1}{2} x_l & - \frac{1}{2} s_2 \\
x_r & = &100 &  & - s_2  \\ \hline
s_1 & = &90 & - x_l &  - s_2 \\
\delta_{x_m}^+ & = & 0 & + \frac{1}{2} x_l & - \frac{1}{2} s_2 & +
	\delta_{x_m}^- \\
\delta_{x_l}^- & = & 30 & - x_l & & & + \delta_{x_l}^+ \\
\delta_{x_r}^+ & = & 10 &&   - s_2  &&& + \delta_{x_r}^-
\end{array}
$$
\end{trivlist}
The basic feasible solution corresponding to this
basic feasible solved form is 
$\{x_m \mapsto 50, s_1 \mapsto 90, x_r \mapsto 100, \delta_{x_m}^+ \mapsto 0
\delta_{x_l}^- \mapsto 30, \delta_{x_r}^+ \mapsto 10, 
x_l \mapsto 0, x_l \mapsto 0, s_2 \mapsto 0, 
\delta_{x_m}^- \mapsto 0,\delta_{x_l}^+ \mapsto 0,\delta_{x_r}^- \mapsto
0\}$
or restricting to the original variables 
$\{x_m \mapsto 50, x_r \mapsto 100, x_l \mapsto 0\}$.
The value of the objective function with this solution is 60.

For instance, in our example increasing $s_2$ from $0$
will decrease the value 
of the objective function.
We first attempt to increase $s_2$. 
We must be careful as we cannot increase the
value of $s_2$ indefinitely as this may cause the value associated with some 
other basic variable to become negative. 
We must examine the equations to determine the maximum value we can choose 
for $x_l$ while still maintaining basic feasible solved form.
The first equation $x_m = 50 + \frac{1}{2} x_l - \frac{1}{2} s_2$ 
allows $s_2$ to take at most a value of $100$, 
as if $s_2$ becomes larger than this, then $x_m$
will become negative.
A row with a positive coefficient of $s_2$ would
not restrict $s_2$ since increasing
its value will just increase the value of the variable on the left hand
side.
In general, 
we choose the most restrictive equation, and use it to eliminate $s_2$.
In the case of ties we arbitrarily
break the tie. 
In this example the most restrictive equation is
$\delta_{x_m}^+  =  0  + \frac{1}{2} x_l  - \frac{1}{2} s_2  +
	\delta_{x_m}^-$
 Writing $s_2$ as the subject we obtain
$s_2 = 0 + x_l - 2 \delta_{x_m}^+ + 2\delta_{x_m}^-$\@.
We replace $s_2$ everywhere by 
$0 + x_l - 2 \delta_{x_m}^+ + 2\delta_{x_m}^-$
and obtain
\begin{quote}\vspace*{-1ex}
minimize $20 - 2 x_l + 1002 \delta_{x_m}^+ + 998 \delta_{x_m}^-  + 2
\delta_{x_l}^+ + 2\delta_{x_r}^- $ 
subject to 
$$
\begin{array}{rlrrrrrr} 
x_m & = &50 &  & + \delta_{x_m}^+ & - \delta_{x_m}^- \\
x_r & = &100 & - x_l &  + 2 \delta_{x_m}^+ & - 2\delta_{x_m}^-  \\ \hline
s_1 & = &90 & - x_l & + 2 \delta_{x_m}^+ & - 2\delta_{x_m}^- \\
s_2 & = &0 & -2 x_l& - 2 \delta_{x_m}^+ &+2\delta_{x_m}^- \\
\delta_{x_l}^- & = & 10 & - x_l & & & + \delta_{x_l}^+ \\
\delta_{x_r}^+ & = & 10 & - x_l & + 2 \delta_{x_m}^+  & - 2\delta_{x_m}^- && +\delta_{x_r}^-
\end{array}
$$
\end{quote}\vspace{-0.9ex}


This step is called \emph{pivoting},
as we have moved $\delta_{x_m}^+$ 
out of the set of basic variables and replaced it by $s_2$.

We continue this process. 
Increasing the value of $x_l$ will decrease 
the value of the objective. 
Note that decreasing $\delta_{x_m}^+$ 
will also decrease the objective function value,
but as $\delta_{x_m}^+$ is constrained to be non-negative, 
it already takes its minimum 
value of $0$ in the associated basic feasible solution.

$x_l$ enters the basis and the 5th and 6th equations are the most
restrictive,
choosing the 5th equation we replace $x_l$
by $10 + \delta_{x_l}^+ - \delta_{x_l}^-$ obtaining the problem
} %%%%%%%%%%%%%%%%%%%%%%%%%%%%%%%%%%%%%%%%%%%%%%%%%%%%%%%%%%%%%%%%%%%%

An optimal solution of this problem can be found using the simplex algorithm,
and results in a tableau
\begin{trivlist}\item
minimize $10 + 1002 \delta_{x_m}^+ + 998 \delta_{x_m}^-  + 2
\delta_{x_l}^- + 2\delta_{x_r}^- $ 
subject to 
$$
\begin{array}{rlrrrrrr} 
x_m & = &50 &  + \delta_{x_m}^+ & - \delta_{x_m}^- \\
x_r & = &70 & + 2 \delta_{x_m}^+ & - 2\delta_{x_m}^- &
		- \delta_{x_l}^+ & + \delta_{x_l}^- \\ \hline
x_l & = & 30  & & & + \delta_{x_l}^+ & - \delta_{x_l}^- \\
s_1 & = &30 &  + 2 \delta_{x_m}^+ & - 2\delta_{x_m}^- &
		-2 \delta_{x_l}^+ & +2 \delta_{x_l}^- \\
s_2 & = &30 &   - 2 \delta_{x_m}^+ &+2\delta_{x_m}^- &
		+ \delta_{x_l}^+ & - \delta_{x_l}^- \\
\delta_{x_r}^+ & = & 10 & + 2 \delta_{x_m}^+  & - 2\delta_{x_m}^- &
	- \delta_{x_l}^+ & + \delta_{x_l}^- & +\delta_{x_r}^-
\end{array}
$$
\end{trivlist}
This corresponds to the solution
$\{x_m \mapsto 50, x_l \mapsto 30, \linebreak x_r \mapsto 70\}$ 
illustrated in Figure~\ref{fig:pict}.
Notice that the weak constraint on $x_r$ is not satisfied.

\ignore{ %%%%%%%%%%%%%%%%%%%%%%%%%%%%%%%%%%%%%%%%%%%%%%%%%%%%%%
The form of the problem is a conjunction of required
inequalities $C$
all of whose variables are constrained to be non-negative
and together with a set of strong edit constraints on 
variables $E$
and a set of weak stay constraints on variables $S$\@.

\textbf{already here}
The handling of the non-required constraints is as follows:
Suppose $x = \alpha$ is a edit or stay constraint.
We introduce two 
error variables $\delta_x^+$ and $\delta_x^-$ that 
are respectively the amount $x$ is  above $\alpha$
or zero if it is equal or below $\alpha$, and 
analogously the amount $x$ is below $\alpha$.
A new equation constraint is introduced: 
$x = \alpha + \delta_x^+ - \delta_x^-$\@.
Because this is the only constraint that involves variables
$\delta_x^+$ and $\delta_x^-$ they will always in any equation in the tableau 
be present with coefficients
$a^+$ and $a^-$ such that $a^+ = -a^-$.


The optimization function includes a term 
$f(strength) * \delta_x^+ + f(strength) * \delta_x^-$
where $f$ is a function that maps 
a strength $strength$
to a real multiple, so that the stronger constraints
are of more importance than the weaker ones.
} %%%%%%%%%%%%%%%%%%%%%%%%%%%%%%%%%%%%%%%%%%%%%%%%%%


\ignore{ %%%%%%%%%%%%%%%%%%%%%%%%%%%%%%%%%%%%%%%%%%%%%%%%%%%%%%
In general solving the optimization problem results in a tableau in 
basic feasible solved form
\begin{quote}\vspace*{-1ex}
minimize $e + \Sigma_{j=1}^m d_j y_j$ subject to
$$\begin{array}{rcl}
\bigwedge_{i=1}^{n} x_i = c_i + \Sigma_{j=i}^m a_{ij} y_j
\end{array}$$
\end{quote}\vspace{-0.9ex}
Each of the constants $c_i$ is non-negative.
Since the tableau represents an optimal solution, 
that minimizes the
objective function it must be the case 
that $d_j \geq 0$ for $1 \leq j \leq m$.
}


\subsection{Incrementality: Resolving the Optimization Problem}
\label{resolving}

Now suppose the user moves the mouse (which
is editing $x_m$) to $x=60$.
We wish to solve a new problem, with
constraints {\strength strong} $x_m = 60$, and
{\strength weak} $x_l = 30$ and {\strength weak} $x_r = 70$
(so that $x_l$ and $x_r$ should stay where they are if possible).
There are two steps.  First, we modify the tableau to reflect the new
constraints we wish to solve.  Second, we resolve the optimization problem
for this modified tableau.

Let us first examine how to modify the tableau to reflect the new values of
the stay constraints.  This will not require reoptimizing the tableau,
since we know that the new stay constraints are satisfied exactly.
Suppose the previous stay value for variable $v$ was $\alpha$, and in the
current solution $v$ takes value $\beta$\@.  The current tableau contains the
information that $v = \alpha + \delta_v^+ - \delta_v^-$, and we need to
modify this so that instead $v = \beta + \delta_v^+ - \delta_v^-$\@.  There
are two cases to consider: (a) both $\delta_v^+$ and $\delta_v^-$ are
parameters, or (b) one of them is basic.

In case (a) $v$ must take the value $\alpha$
in the current solution since both $\delta_v^+$ and 
$\delta_v^-$ take the value
$0$ and $v = \alpha + \delta_v^+ - \delta_v^-$\@.
Hence $\beta = \alpha$ and no changes need to be made.

In case (b) assume without loss of generality that $\delta_v^+$ is
basic.  In the original equation representing the stay constraint, the
coefficient for $\delta_v^+$ is the negative of the coefficient for
$\delta_v^-$\@.  Since these variables occur in no other constraints, this
relation between the coefficients will continue to hold as we perform
pivots.  In other words, $\delta_v^+$ and $\delta_v^-$ come in pairs: any
equation that contains $\delta_v^+$ will also contain $\delta_v^-$ and vice
versa.  Since $\delta_v^+$ is assumed to be basic, it occurs exactly once
in an equation with constant $c$, and further this equation also contains
the only occurrence of $\delta_v^-$\@.  In the current solution $\{v \mapsto
\beta, \delta_v^+ \mapsto c, \delta_v^- \mapsto 0\}$, and since the
equation $v = \alpha + \delta_v^+ - \delta_v^-$ holds, $\beta = \alpha +
c$\@.  To replace the equation $v = \alpha + \delta_v^+ - \delta_v^-$ by $v =
\beta + \delta_v^+ - \delta_v^-$ we simply need to replace the constant $c$
in this row by $0$\@.  Since there are no other occurrences of $\delta_v^+$
and $\delta_v^-$ we have replaced the old equation with the new.

For our example, to update the tableau for the new values for the stay
constraints on $x_l$ and $x_r$ we simply set the constant of last equation
(the equation for $\delta_{x_r}^+$) to 0.

Now let us consider the edit constraints.  Suppose the previous edit value
for $v$ was $\alpha$, and the new edit value for $v$ is $\beta$\@.  The
current tableau contains the information that \linebreak
$v = \alpha + \delta_v^+ - \delta_v^-$, 
and again we need to modify this so that instead $v = \beta +
\delta_v^+ - \delta_v^-$\@.  To do so we must replace every occurrence of
$\delta_v^+ - \delta_v^-$ by $\beta - \alpha + \delta_v^+ - \delta_v^-$,
taking proper account of the coefficients of $\delta_v^+$ and $\delta_v^-$\@.
(Again, remember that $\delta_v^+$ and $\delta_v^-$ come in pairs.)

If either of $\delta_v^+$ and $\delta_v^-$ is basic, this simply involves
appropriately modifying the equation in which they are basic.  Otherwise, if
both are non-basic, then we need to change every equation of the form \\
\hspace*{5mm}$x_i = c_i + a'_v \delta_v^+ - a'_v \delta_v^- + e$ \\
to \\
\hspace*{5mm}
$x_i = c_i + a'_v (\beta - \alpha) + a'_v \delta_v^+ - a'_v \delta_v^- + e$ \\
Hence modifying the tableau to reflect the new values of edit and stay
constraints involves only changing the constant values in some equations.
The modifications for stay constraints always result in a tableau in basic
feasible solved form, since it never makes a constant become negative.
In contrast the modifications for edit constraints may not.

To return to our example, suppose we pick up $x_m$ with the mouse and
move it to 60.  Then we have that $\alpha = 50$ and $\beta = 60$,
so we need to add 10 times the 
coefficient of $\delta_{x_m}^+$ to the constant part of every row.
The modified tableau, after the updates for both the stays and edits, is 
\begin{trivlist}\item
minimize $20 + 1002 \delta_{x_m}^+ + 998 \delta_{x_m}^-  + 2
\delta_{x_l}^- + 2\delta_{x_r}^- $ 
subject to 
$$
\begin{array}{rlrrrrrr} 
x_m & = &60 &  + \delta_{x_m}^+ & - \delta_{x_m}^- \\
x_r & = &90 & + 2 \delta_{x_m}^+ & - 2\delta_{x_m}^- &
		- \delta_{x_l}^+ & + \delta_{x_l}^- \\ \hline
x_l & = & 30  & & & + \delta_{x_l}^+ & - \delta_{x_l}^- \\
s_1 & = &50 &  + 2 \delta_{x_m}^+ & - 2\delta_{x_m}^- &
		-2 \delta_{x_l}^+ & +2 \delta_{x_l}^- \\
s_2 & = &10 &   - 2 \delta_{x_m}^+ &+2\delta_{x_m}^- &
		+ \delta_{x_l}^+ & - \delta_{x_l}^- \\
\delta_{x_r}^+ & = & 20 & + 2 \delta_{x_m}^+  & - 2\delta_{x_m}^- &
	- \delta_{x_l}^+ & + \delta_{x_l}^- & +\delta_{x_r}^-
\end{array}
$$
\end{trivlist}
Clearly it is feasible and already in optimal form, and so we have
incrementally resolved the problem by simply modifying constants in the
tableaux. The new tableaux give the solution
$\{x_m \mapsto 60, x_l \mapsto 30, x_r \mapsto 90\}$\@.
So sliding the midpoint rightwards 
has caused the right point to slide rightwards as well, but twice as far.
The resulting diagram is shown at the top of Figure~\ref{fig:quasi}.

\begin{figure}[htb]
\begin{center}
\input quasi.eepic
\end{center}
\caption{Resolving the constraints\label{fig:quasi}}
\end{figure}

Suppose we now move $x_m$ from 60 to 90.  
The modified tableau is 
\begin{trivlist}\item
minimize $60 + 1002 \delta_{x_m}^+ + 998 \delta_{x_m}^-  + 2
\delta_{x_l}^- + 2\delta_{x_r}^- $ 
subject to 
$$
\begin{array}{rlrrrrrr} 
x_m & = &90 &  + \delta_{x_m}^+ & - \delta_{x_m}^- \\
x_r & = &150 & + 2 \delta_{x_m}^+ & - 2\delta_{x_m}^- &
		- \delta_{x_l}^+ & + \delta_{x_l}^- \\ \hline
x_l & = & 30  & & & + \delta_{x_l}^+ & - \delta_{x_l}^- \\
s_1 & = &110 &  + 2 \delta_{x_m}^+ & - 2\delta_{x_m}^- &
		-2 \delta_{x_l}^+ & +2 \delta_{x_l}^- \\
s_2 & = &-50 &   - 2 \delta_{x_m}^+ &+2\delta_{x_m}^- &
		+ \delta_{x_l}^+ & - \delta_{x_l}^- \\
\delta_{x_r}^+ & = & 60 & + 2 \delta_{x_m}^+  & - 2\delta_{x_m}^- &
	- \delta_{x_l}^+ & + \delta_{x_l}^- & +\delta_{x_r}^-
\end{array}
$$
\end{trivlist}
The tableau is no longer in basic feasible solved form,
since the constant of the row 
for $s_2$ is negative, even though $s_2$ is supposed to be non-negative.

Thus, in general, after updating the constants for the edit constraints,
the simplex tableau $C_S$ may no longer be in basic feasible solved form,
since some of the constants may be negative.  However, the tableau is still
in basic form, so we can still
read a solution directly from it.
% as before.
% ! to avoid 1 line in the paragraph!
And since no coefficient has changed, in particular in the optimization
function, the resulting tableau reflects an optimal but not feasible solution.

We need to find a feasible and optimal solution.  We could do so by adding
artificial variables (as we did when adding a constraint), optimizing the
sum of the artificial variables to find an initial feasible solution, and
then reoptimizing the original problem.
But we can do much better.  The process of moving from an optimal and
infeasible solution to an optimal and feasible solution is exactly the dual
of normal simplex algorithm, where we progress from a feasible and
non-optimal solution to feasible and optimal solution.  Hence we can use
the \emph{dual simplex algorithm} to find a feasible solution while staying
optimal.

Solving the dual optimization problem starts from
an infeasible optimal tableau of the form
\begin{quote}\vspace*{-1ex}
minimize $e + \Sigma_{j=1}^m d_j y_j$ subject to
$$\begin{array}{rcl}
\bigwedge_{i=1}^{n} x_i = c_i + \Sigma_{j=i}^m a_{ij} y_j
\end{array}$$
\end{quote}\vspace{-0.9ex}
where some $c_i$ may be negative 
for rows with non-negative basic variables (infeasibility) 
and each $d_j$ is non-negative (optimality).

The dual simplex algorithm selects an exit variable
by finding a row $I$ with non-negative basic variable
$x_I$ and negative constant $c_I$\@.  
The entry variable is the variable $y_J$
such that
the ratio $d_J/a_{IJ}$ is the minimum of all $d_j/a_{Ij}$
where $a_{Ij}$ is positive. This ensures that when pivoting we stay at an
optimal solution.
The pivot, replacing $y_j$ by \\
\hspace*{5mm} 
$-1/a_{Ij} (-x_I + c_I + \Sigma_{j=1, j\neq J }^m a_{Ij} y_j)$ \\
is performed as in the (primal) simplex algorithm.

\ignore{
\begin{figure}[tb]
\begin{center}
\fbox{
\begin{minipage}{10cm}
\begin{tabbing}
xx \= xx \= xx \= xx \= \kill
%Boolean
\textsf{re\_opt}($C_S$,$f$) \\
\> \textbf{foreach} $stay:v \in C$ \\
\> \> \textbf{if} $\delta_v^+$ or $\delta_v^-$ is basic in row $i$ \textbf{then} $c_i$ :=
0 \textbf{endif} \\
\>  \textbf{endfor} \\
\> \textbf{foreach} $edit:v \in C$ \\
\> \> \textbf{let} $\alpha$ and $\beta$ be the previous and current edit
values for $v$\\
\> \> \textbf{let} $\delta_v^+$ be $y_j$ \\
\> \> \textbf{foreach} $i \in \{1, \ldots ,n\}$ \\
\> \> \>  $c_i$ := $c_i + a_{ij} (\beta - \alpha)$ \\
\> \> \textbf{endfor} \\
\> \textbf{endfor} \\
\> \textbf{repeat} \\
%\> \> \textbf{if} for each $i \in \{ 1, \ldots , n\}$ 
%	$c_i \ge 0$ or $x_I \geq 0 \not\in C_I$  \textbf{then} \\
\> \> \% Choose variable $x_I$ to become non-basic\\
\> \> \textbf{choose} $I$ where $c_I < 0$ \\
\> \> \textbf{if} there is no such $I$ \\
\> \> \> \textbf{return} $true$  \\
\> \> \textbf{endif} \\

\> \> \% Choose variable $y_J$ to become basic\\
\> \> \textbf{if} for each $j \in \{ 1, \ldots , m\}$ $a_{Ij} \leq 0$ \textbf{then} \\
\> \> \> \textbf{return} $false$\\
\> \> \textbf{endif} \\
\> \> \textbf{choose} $J \in \{1, \ldots , m\}$ such that\\
\> \> \> $d_J / a_{IJ} =
    \min_{j \in \{1, \ldots , m\}} \{ d_j / a_{Ij} \mid a_{Ij} > 0 \}$\\
\> \> $e$ := $( x_I - c_I - \sum_{j=1, j\ne J}^m a_{Ij} y_j)/ a_{IJ}$\\
\> \> replace $y_J$ by $e$ in $f$\\
\> \> \textbf{for} each $i \in \{1, \ldots ,n\}$\\
\> \> \> \textbf{if} $i \ne I$ \textbf{then}
       replace $y_J$ by $e$ in row $i$ \textbf{endif}\\
\> \> \textbf{endfor}\\
\> \> replace the $I^{th}$ row by $y_J = e$ \\
\> \textbf{until} $false$ \\
\end{tabbing}
\end{minipage}
}
\end{center}
\caption{Dual Simplex Re-optimization\label{fig:simplex-dual}}
\end{figure}
}

Continuing the example above
we select the exit variable $s_2$, 
the only non-negative basic variable for a row with negative constant.  
We find that $\delta_{x_l}^+$
has the minimum ratio since its coefficient in the optimization function
is 0, so it will be the entry variable.
Replacing $\delta_{x_l}^+$ everywhere by 
$50 + s_2 + 2 \delta_{x_m}^+ - 2 \delta_{x_m}^- + \delta_{x_l}^+$
we obtain the tableau

\begin{trivlist}\item
minimize $30060 + 1002 \delta_{x_m}^+ + 998 \delta_{x_m}^-  + 2
\delta_{x_l}^- + 2\delta_{x_r}^- $ 
subject to 
$$
\begin{array}{rlrrrrrr} 
x_m & = &90 & & + \delta_{x_m}^+ & - \delta_{x_m}^- \\
x_r & = &100 & - s_2 \\ \hline
x_l & = & 80 & + s_2 & + 2 \delta_{x_m}^+ & - 2 \delta_{x_m}^- \\
s_1 & = &110 & - 2 s_2 & + 2 \delta_{x_m}^+ & - 2\delta_{x_m}^- \\
\delta_{x_l}^+ & = & 50 & + s_2 & + 2 \delta_{x_m}^+ & - 2\delta_{x_m}^- &
		+ \delta_{x_l}^- \\
\delta_{x_r}^+ & = & 40 & - s_2 &&&&  +\delta_{x_r}^-
\end{array}
$$
\end{trivlist}
The tableau is feasible (and of course still
optimal) and represents the solution
$\{x_m \mapsto 90, x_r \mapsto 100, x_l \mapsto 80\}$\@.
So by sliding the midpoint further right, the rightmost point hits the wall
and the left point slides right to satisfy the constraints.
The resulting diagram is shown at the bottom of Figure~\ref{fig:quasi}.

To summarize, incrementally finding a new solution for new input variables
involves updating the constants in the tableaux to reflect the updated stay
constraints, then updating the constants to reflect the updated edit
constraints, and finally reoptimizing if needed.  In an interactive
graphical application, when using the dual optimization method typically a
pivot is only required when one part first hits a barrier, or first moves
away from a barrier.  The intuition behind this is that
when a constraint
first becomes unsatisfied, the value of one of its error variables will
become non-zero, and hence the variable will have to enter the basis;
when a constraint first becomes satisfied,
we can move one of its error variables out of the basis.

In the example, pivoting occurred when the right point $x_r$ came up against a
barrier.  Thus, if we picked up the midpoint $x_m$ with the mouse and
smoothly slid it rightwards, 1 pixel every screen refresh, only one pivot
would be required in moving from 50 to 95.  This illustrates why the dual
optimization is well suited to this problem and leads to efficient
resolving of the hierarchical constraints.

\section{QOCA: QUADRATIC OPTIMIZATION}
\label{quadratic}

Another useful way of comparing solutions to constraint hierarchies is
least-squares-better, in which case we are interested in
solving optimization problems of the following form, referred to as $QP$:
\begin{quote}
minimize $f$ subject to $C$  \\
where $f = \sum_{i=1}^n w_i (x_i - d_i)^2$
\end{quote}\vspace{-0.9ex}
The variables are $x_1, \ldots , x_n$, and $C$  is the set of
required constraints.  The desired value for variable
$x_i$ is $d_i$, and the ``weight'' associated with that desire (which
should reflect the hierarchy) is $w_i$\@.

This problem is a type of {\em quadratic programming} in which a quadratic
optimization function is minimized with respect to a set of linear
arithmetic equality and inequality constraints.  In particular, since the
optimization function is a sum of squares, the problem is an example
of {\em convex} quadratic programming, meaning that the local minimum is
also the global minimum.  This is fortunate, since convex quadratic
programming has been well-studied and efficient methods for solving these
problems are well-known in the operations research community.

\subsection{Active Set Method}
\label{active-sets}

Our implementation of QOCA uses the {\em active set method}
\cite{fletcher-book} to solve the convex quadratic programming problem.
This method is an iterative technique for solving constrained optimization
problems with inequality constraints.  It is reasonably robust and quite
fast, and is the method of choice for medium scale problems consisting of
up to 1000 variables and constraints.

The key idea behind the algorithm is to solve a sequence of constrained
optimization problems $O_0$, ..., $O_t$\@.  
Each problem minimizes $f$ with
respect to a set of equality constraints, ${\cal A}$, called the {\em
active set}.  The active set consists of the original equality constraints
plus those inequality constraints that are ``tight,'' in other words, those
inequalities that are currently required to be satisfied as
equalities. The other inequalities are ignored for the moment.

%Note that the active set method is closely related to the simplex method. Those
%inequalities whose slack variables are not basic are in the active set,
%while those whose slack variables are basic are not.  Pivoting corresponds
%to moving one inequality out of the active set and replacing it by another.

Essentially, each optimization problem $O_i$ can be treated as an unconstrained
quadratic optimization problem, denoted by $U_i$\@. To obtain $U_i$, we
rewrite the equality constraints in $O_i$ in basic feasible solved
form, and then eliminate all basic variables in the objective function $f$\@.
The optimal solution is the point at which all of the partial derivatives of 
$f$ equal zero. The problem $U_i$ can be solved easily, since we are 
dealing with a convex quadratic function $f$ and so 
its derivatives are linear. 
As a result, to solve $U_i$ we need only solve a system of
linear equations over unconstrained variables. 

In more detail,
in the active set method, we assume at each stage
that a feasible initial guess 
$\vx_0 = (x_1, \cdots x_n)^T$ is available, as well as the corresponding
active set ${\cal A}$\@. 
Assume that we have just solved the optimization problem $O_0$, and let its 
solution be $\vx^\ast_0$\@. We  face the following two possibilities 
when determining the new approximate solution $\vx_1$\@. 
\bl{\arabic{bean}.}
\item $\vx^\ast_0$ is feasible with respect
 to the constraints in $O_0$ but it
violates some inequality constraints in $QP$ that are not in the current
active set ${\cal A}$\@. In this case, a scalar $\alpha \in [0,1]$ is 
selected, such that it is as large as possible and the point $\vx_0 + \alpha
(\vx^\ast_0 - \vx_0) $ is feasible. This point is taken as the new approximate
solution $\vx_1$, and the violated constraints are added to the active
set, giving rise to a new optimization problem $O_1$\@. 
\item $\vx^\ast_0$ is feasible with respect to the original problem $QP$\@.
It is directly taken as the new 
approximate solution $\vx_1$ and we test to see it is  also
optimal $QP$\@. This requires us to 
check if there exists a direction $\vs$ at $\vx_1$, such that a feasible
incremental step along $\vs$ reduces $f$\@. If such direction $\vs$ exists,
then one constraint is taken out of the active set ${\cal A}$ 
to generate the direction $\vs$, which results in a new 
optimization problem $O_1$\@. If no such direction exists we are finished
since $\vx_1$ is both feasible and optimal.
\el
%
If the active set is modified, the whole process is repeated until 
the optimal solution is reached.
%
%*** example ***
%

Consider our working example with the weak constraints that
$x_m=50$, $x_l=30$ and $x_r=70$\@. This gives rise to the
minimization problem $QP_1:$
\begin{quote}
minimize $f_1 = (x_m - 50)^2 + (x_l - 30)^2 
+ (x_r - 70)^2 $ subject to
$$
\begin{array}{rrrrcl}
(1)  & 2 x_m & -x_l & -x_r & = & 0 \\
(2)      &       & -x_l & +x_r & \geq & 10 \\
(3)      &       &      & -x_r & \geq & -100 \\
(4)      &       & x_l  &      & \geq & 0 
\end{array}
$$
\end{quote}
Although it is obvious that $x_m = 50, x_l = 30, x_r = 70 $
or $ \vx^\ast = (50,30,70)^T$ is the optimal solution, it is still
instructive to see how the active set method computes this. 
The initial guess and active set are read from the augmented simplex form
tableux. We start with
an initial guess $x_m = 50, x_l = 0, \linebreak x_r = 100$,
i.e.\ $\vx_0=(50,0,100)^T$,
and constraints 1, 3 and 4 are active. Thus 
${\cal A}^{(1)}_0 = \{ 1,3,4 \}$ is the initial active set. The equality
constrained optimization problem $O^{(1)}_0$ is therefore
\begin{quote}
minimize $f_1$ subject to
$$
\begin{array}{rrrrrr}
          & 2x_m & -x_l & -x_r & = & 0 \\
	  &      &      & -x_r & = & -100 \\
	  &      & x_l  &      & = & 0
\end{array}
$$
\end{quote}
The problem $O^{(1)}_0$ has only one feasible solution $x_m = 50, \linebreak
x_l = 0,
x_r = 100$, so it is also the optimal solution, denoted by $\vx^\ast_0$\@.
Next we check if $\vx^\ast_0$ is the optimal solution to the problem
$QP_1$\@. Constraint 4 forces $x_l$ to take the value 0
in $\vx^\ast_0$\@. However, the value of the
objective function $f_1$ can be reduced if $x_l$ 
is increased. 
Thus the 4th constraint $\vx_l \geq 0 $ can be moved out of the active
set in order to further reduce the value of $f_1$\@. This gives
$\vx_1 = \vx^\ast_0$ as the new approximate
solution, ${\cal A}^{(1)}_1 = \{1,3\}$ as the active set and the optimization
problem $O^{(1)}_1$ as 
\begin{quote}
minimize $f_1$ subject to
$$\begin{array}{rrrrrr} 
          & 2x_m & -x_l & -x_r & = & 0 \\
	  &      &      & -x_r & = & -100
\end{array}
$$
\end{quote}
To solve $O^{(1)}_1$, we rewrite the constraints in $O^{(1)}_1$ to a basic
feasible solved form $x_r = 100 \wedge x_l = 2x_m - 100$, and then eliminate 
basic variables in the function $f_1$\@. This results in the following 
unconstrained optimization problem 
\begin{quote}
minimize $(x_m - 50)^2 + (2x_m - 100 -30)^2 + (100 - 70)^2$
\end{quote}
Setting the derivative to be zero we obtain 
$$ 2(x_m -50) + 2 \times 2 (2x_m - 130) = 0.$$
Solving this together with the constraint in $O^{(1)}_1$, the optimal solution
of $O^{(1)}_1$ is found to be $\vx^\ast_1 = (62,24,100)^T$\@. It is
easy to verify that $\vx^\ast_1$ is still feasible. Similarly to the 
case for $\vx^\ast_0$, in $\vx^\ast_1$
$x_r$ is forced to take the value 100 because of the
3rd constraint, yet the function value $f_1$ can be
reduced if $x_r$ is decreased. 
So the 3rd constraint $-x_r \geq -100$ is moved out of the active
set. We now have the new approximate solution $\vx_2 = \vx^\ast_1$, 
the active set 
${\cal A}^{(1)}_2 = \{ 1 \}$ and the optimization problem $O^{(1)}_2$: \\
$$
\mbox{minimize $f_1$ subject to $2 x_m - x_l - x_r = 0.$}
$$
To solve this problem, we repeat the same procedure as for solving $O^{(1)}_1$\@.
%Rewrite the constraint as $x_l = 2x_m - x_r$ and substitute it into the 
%function $f_1$, then we have the unconstrained problem \\
%\begin{quote}
%\hspace*{0.5 cm} $ \min \ q = (x_m-50)^2 + (x_l-30)^2 + (2x_m-x_l-70)^2$ \\
%\end{quote}\vspace{-0.9ex}
The solution to this problem satisfies the equations:
\bel{Q1_2}
\begin{array}{lclcl}
2(x_m-50) &+& 2 \times 2(2x_m - x_l -70)&  = & 0  \\
2(x_l-30) &+& 2(2x_m - x_l -70) & = & 0  
\end{array}  
\eel
%which is obtained by setting all partial derivatives of $q$ to be zero.  
These together with the constraint in $O^{(1)}_2$ have the solution
$\vx^\ast= (50,30,70)^T$\@. This is the optimal solution to $O^{(1)}_2$ and
is also the optimal solution to the original problem $QP_1$\@. 


\begin{figure}[htb]
\begin{center}
\input qoca.eepic
\end{center}
\caption{Resolving the constraints using QOCA\label{fig:qoca}}
\end{figure}


Now imagine that we have started to manipulate the diagram.
We have the weak constraints that $x_l = 30$ and $x_r = 70$ and
the strong constraint that $x_m = 60$\@.
Reflecting this, we change the first term in the function $f_1$ to be
$1000(x_m-60)^2$, denote it as $f_2$ and the corresponding optimization
problem as $QP_2$\@. 
Starting from $\vx_0 = (50,30,70)^T$, which is the optimal solution
to $QP_1$, an equality constrained problem $O^{(2)}_0$ is formed. $O^{(2)}_0$
is the same as $O^{(1)}_2$, except that they have different objective 
functions. The solution to $O^{(2)}_0$ satisfies similar linear equations
to those of (\ref{Q1_2}). These can be obtained by replacing the term $2(x_m-50)$
in the first equation of (\ref{Q1_2}) by $1000 (x_m-60)$
reflecting the change in the objective function. A solved form 
for these equations is 
\bel{Q2_0}
\begin{array}{lclcl}
x_m & = & \frac{500}{501} \times 60 & + & \frac{50}{501} \\
x_l & = & x_m & - & 20 
\end{array}
\eel
which leads to the optimal solution for both $O^{(2)}_0$ and $QP_2$ 
as $x_m = 59.98, x_l = 39.98, x_r = 79.98 $\@. 
(The exact least-squares-better
solution is actually $x_m = 60, x_l = 40, x_r = 80$\@.  With
quadratic optimization the strong constraints don't completely dominate the
weak ones in the computed solution.  However, by choosing a suitably large
constant we found a solution that {\em is} least-squares-better to under
a one-pixel resolution, so that the deviation from a least-squares-better
solution would not be visible in an interactive system.  (See
\cite{borning-simplex-tr} for more on this issue.)

To modify the active set method so that it is incremental for
resolving, we observe that 
changing the desired variable values only changes the
optimization function $f$\@. Thus we can reuse the active set from the last
resolve and reoptimize with respect to this.  In most cases the active set does
not change, and so we are done.  Otherwise we proceed as above.

For example, if we now move $x_m$ from 60 to 90,
we change the objective function again, but need only change the 
desired values and can keep the weights the same as they are in $f_2$,
e.g.\ in the new objective function $f_3$, the variable $x_m$ has 
a new desired value 90. The corresponding optimization problem
is referred to as $QP_3$\@. To solve this problem, the {\em resolve}
procedure makes use of the information from the
previous solve $QP_2$, while applying the active set method to $QP_3$\@. 
When resolving, it is important to notice that, if we start from 
the solution for the previous problem $QP_2$, i.e.\ $\vx_0 = (59.98,
39.98,79.98)^T$, then the solution to the corresponding equality
constrained problem $O^{(3)}_0$,
\begin{quote}
minimize $f_3 $ subject to $2x_m -x_l -x_r = 0$,
\end{quote}
can be easily obtained. In fact, one can just replace the desired
value 60 for $x_m$ in (\ref{Q2_0}) by its new desired value 90, which 
leads to the optimal solution to $O^{(0)}_3$ as 
$\vx^\ast_0 = (89.9202, 69.9202,109.9202)^T$\@. If the desired value 
does not change too much, it is quite likely that $\vx^\ast_0$ is also 
optimal for $QP_3$\@. Unfortunately, this is not the case for this example,
since $\vx^\ast_0$ violates the 3rd constraint $-x_r \geq -100$\@.
Choosing $\alpha \in [ 0, 1]$ to be as big as possible while still ensuring
that
$\vx_1 = \vx_0 + \alpha(\vx^\ast_0 - \vx_0)$ is feasible, 
we have $\alpha = 0.6687$
and $\vx_1 = \vx_0 + \alpha(\vx^\ast_0 - \vx_0)$ as the new approximate 
solution, at which the 3rd constraint becomes active. By
solving the corresponding
equality constrained problem $Q^{(3)}_1$,
\begin{quote}
minimize $f_3 $ subject to $2x_m - x_l -x_r = 0, 
\ -x_r = -100 $,
\end{quote}
the optimal solution to $QP_3$ is found to be 
$x_m = 89.9003$, $x_l = 79.8007$,
$x_r = 100$. 

Figure~\ref{fig:qoca} shows the effect of moving the horizontal line
with the least squares comparator. With this comparator the line
moves right maintaining the same length until it hits the right boundary,
at which point it starts to compress. This contrasts with the behaviour
of the locally-error-better comparator
in which the line grew until it bumped against the side.


\section{EMPIRICAL EVALUATION}
\label{empirical-evaluation}

%Both algorithms have been implemented and tested.

Our algorithms for incremental addition and deletion of equality and inequality
constraints and for solving and resolving for the least-square comparator
using the QOCA algorithm have been implemented as part of the QOCA C++
constraint solving toolkit. 
The results are very satisfactory.
For a test problem with 300 constraints and
300 variables, adding a constraint takes
on average  $1.5$~msec, deleting a constraint $1.6$~msec, the initial
solve  $12$~msec, and subsequent resolving as the point moves $4.5$~msec.
For a larger problem with 900 constraints and variables,
adding a constraint takes
on average  $9.7$~msec, deleting a constraint $17$~msec, the initial
solve  $120$~msec, and subsequent resolving as the point moves $67$~msec.
These tests were run a sun4m sparc, running SunOS~5.4.

Running Cassowary on the same problems, for the 300 constraint problem,
adding a constraint takes on average $38$~msec (including the initial
solve), deleting a constraint $46$~msec, and resolving as the point moves
$15$~msec.  (Stay and edit constraints are represented explicitly in this
implementation, so there were also stay constraints on each variable, plus
two edit constraints, for a total of 602 constraints.)
For the 900 constraint problem, adding a constraint takes on
average $98$~msec (again including the initial solve), deleting a
constraint $151$~msec, and resolving as the point moves $45$~msec.  These
tests were run using an implementation in OTI Smalltalk Version 4.0 running
on a IBM Thinkpad 760EL laptop computer.

As these measurements are for implementations in different languages,
running on different machines, they should not be viewed as any kind of
head-to-head comparison.  Nevertheless, they indicate that both
algorithms are eminently practical for use with interactive graphical
applications.

The QOCA toolkit has been employed in a number of applications.
The first application is part of an intelligent pen and paper
interface that contains a parser to incrementally parse diagrams drawn by
the user using a stylus, and that has a diagram editor that respects
the semantics of the diagram by preserving the constraints recognized
in the parsing process.  QOCA is used for both error correction in
parsing and for diagram manipulation in the editor~\cite{chok-marriott95}.
A second QOCA application is for layout of trees and graphs in the
presence of arbitrary linear arithmetic constraints and with
suggested placements for some nodes~\cite{he-marriott96}.
A Cassowary application currently being developed 
is a web authoring tool
\cite{borning-multimedia-97}, in which the appearance of a page is
determined by constraints from both the web author and the viewer.  

\subsection*{Acknowledgments}

This project has been funded in part by the National Science Foundation
under Grants \mbox{IRI-9302249} and \mbox{CCR-9402551} and by Object
Technology International.  Alan Borning's visit to Monash University and
the University of Melbourne was sponsored in part by the
Australian-American Educational Foundation (Fulbright Commission).


\small
\bibliography{/projects/weird/constraints}
\bibliographystyle{plain}

\end{document}

